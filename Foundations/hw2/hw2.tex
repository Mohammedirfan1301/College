\documentclass[12pt]{article}
\usepackage{amsfonts}
\usepackage{comment}
\usepackage{tikz}
\usetikzlibrary{arrows,automata}
\begin{document}

\noindent
Jason Downing \\
Email: jason\_downing@student.uml.edu \\
Foundations of Computer Science \\
Homework \#2 - Chapter 1: DFA, NFA\\
10/6/2016 \\

\noindent
1.3 \\
The formal description of a DFA M is \\
\{q1, q2, q3, q4, q5 \}, \{u, d\}, $\delta$, q3 , \{q3\}, \\
where $\delta$  is given by the following table:
\begin{center}
\begin{tabular}{ c | c c }
 & u & d \\
\hline
 q1 & q1 & q2 \\ 
 q2 & q1 & q3 \\  
 q3 & q2 & q4 \\
 q4 & q3 & q5 \\
 q5 & q4 & q5
\end{tabular}
\end{center}

\noindent
Give the state diagram of this machine.	\\

\noindent
Initial state is q3, so that is where the machine will start. \\
We can use the table to create the nodes, and connect them similar \\
to HW1's Q0.8 \\

\begin{center}
\begin{tikzpicture}[>=stealth',shorten >=2pt, auto, node distance=2cm]
	% Nodes {q1, q2, q3, q4, q5}
	\node [state] 	(q1) 				{q1};
	\node [state] 	(q2) [right of=q1]  {q2};
	\node [state] 	(q3) [right of=q2]  {q3};
	\node [state]	(q4) [right of=q3]  {q4};
	\node [state]   (q5) [right of=q4]  {q5};

	% Paths	{{1, 2}, {2, 3}, {1, 3}, {2, 4}, {1, 4}}
	\path [->] 	(q1) edge [loop above] node	{d}	(q1)
	\path [->] 	(q1) edge [] node	{u}	(q2)
				
\end{tikzpicture} \\
\end{center}


\noindent
1.4: a, c, e, f, g \\


\noindent
1.5: c, d, e, f, g, h \\


\noindent
1.6: a, b, c, d, e, f, g, h, I, j, k, l, m, n \\


\noindent
1.7: b, c, d, e, g, h \\


\noindent
1.8: a, b \\


\noindent
1.9: a, b \\


\noindent
1.10: a, b, c \\


\noindent
1.12 \\


\noindent
1.13 \\

\noindent
1.16 \\

\noindent
1.17: a, b \\

\noindent
1.18 \\

\noindent
1.20: a, b, c, d, e, f, g, h \\

\noindent
1.21 \\

\noindent
1.22 \\

\end{document}