\documentclass[12pt]{article}
\setlength\parindent{0pt}
\usepackage{amsfonts}
\usepackage{comment}
\usepackage{tikz}
\usepackage{forest}
\usetikzlibrary{arrows,automata}

% Found this on Stackoverflow, seems to make a better explicit space than the default
% "\textvisiblespace" does, so I used \Vtextvisiblespace instead.
% https://tex.stackexchange.com/questions/50804/explicit-space-character
\newcommand\Vtextvisiblespace[1][.8em]{%
	\mbox{\kern.07em\vrule height.3ex}%
	\vbox{\hrule width#1}%
	\hbox{\vrule height.3ex}}

% Also found this on Stackoverflow for making tabs.
% https://tex.stackexchange.com/questions/198432/using-the-tab-command
\newcommand\tab[1][1cm]{\hspace*{#1}}

\begin{document}

\noindent
Jason Downing \\
Email: jason\_downing@student.uml.edu \\
Foundations of Computer Science \\
Homework \# - Chapter 4 + Chapter 5 \\
12/1/2016 \\

%*********************************************************************************
% Extra Credit Documentation!
%*********************************************************************************
**********************************************************************************
Since I started this assignment early, and I had some time left before the assignment
was due, I decided to do the following \textbf{Extra Credit} problems: \\

1. ?? \\

These problems are at the end of my PDF in the "\textbf{Extra Credit}" section.

********************************************************************************** \\

%*********************************************************************************
%*			Chapter 4 starts here!
%*********************************************************************************
% 4.2
\pagebreak
\textbf{4.2} Consider the problem of determining whether a $DFA$ and a regular \\
expression are equivalent. Express this problem as a language and show that it is decidable. \\



%*********************************************************************************
% 4.3
\pagebreak
\textbf{4.3} Let $ALL_{DFA}$ = $\{ \langle A \rangle | A$ is a $DFA$ and $L(A) = \sum^*\}$.
Show that $ALL_{DFA}$ is decidable. \\



%*********************************************************************************
% 4.4
\pagebreak
\textbf{4.4} Let $A\epsilon_{CFG}$ = $\{ \langle G \rangle | G$ is a $CFG$ that generates $\epsilon \}$.
Show that $A\epsilon_{CFG}$ is decidable. \\



%*********************************************************************************
% 4.6
\pagebreak
\textbf{4.6} Let $X$ be the set $\{1, 2, 3, 4 ,5 \}$ and let $Y$ be the set $\{6, 7, 8, 9, 10 \}$.
We describe the functions $f: X \rightarrow Y$ and $g: X \rightarrow Y$ in the following tables.
Answer each part and give a reason for each negative answer. \\

\begin{center}
\begin{tabular}{l|l}
	n & $f(n)$ \\ \hline
	1 & 6      \\
	2 & 7      \\
	3 & 6      \\
	4 & 7      \\
	5 & 6      \\
\end{tabular}
\quad \quad
\begin{tabular}{l|l}
	n & $g(n)$ \\ \hline
	1 & 10     \\
	2 & 9      \\
	3 & 8      \\
	4 & 7      \\
	5 & 6      \\
\end{tabular}
\end{center}


\textbf{a.} Is $f$ one-to-one? \\

\textbf{b.} Is $f$ onto? \\

\textbf{c.} Is $f$ a correspondence? \\

\textbf{d.} Is $g$ one-to-one? \\

\textbf{e.} Is $g$ onto? \\

\textbf{f.} Is $g$ a correspondence? \\

%*********************************************************************************
% 4.7
\pagebreak
\textbf{4.7} Let $\mathcal{B}$ be the set of all infinite sequences over $\{0, 1\}$. 
Show that $\mathcal{B}$ is uncountable using a proof by diagonalization. \\



%*********************************************************************************
% 4.8
\pagebreak
\textbf{4.8} Let $T = \{(i, j, k)| i, j, k \; \in \; \mathcal{N}\}$. Show that $T$ is countable. \\



%*********************************************************************************
%*			Chapter 5 starts here!
%*********************************************************************************

%*********************************************************************************
% 5.1
\pagebreak
\textbf{5.1} \\



%*********************************************************************************
% 5.2
\pagebreak
\textbf{5.2} \\



%*********************************************************************************
% 5.3
\pagebreak
\textbf{5.3} \\



%*********************************************************************************
% 5.4
\pagebreak
\textbf{5.4} \\




%*********************************************************************************
% Extra Credit Section!
%*********************************************************************************
\pagebreak
**********************************************************************************

\tab\tab\tab \textbf{EXTRA CREDIT SECTION BEGINS HERE}

**********************************************************************************



\end{document}
