\documentclass[12pt]{article}
\setlength\parindent{0pt}
\usepackage{amsfonts}
\usepackage{comment}
\usepackage{tikz}
\usepackage{forest}
\usetikzlibrary{arrows,automata}

% Found this on Stackoverflow, seems to make a better explicit space than the default
% "\textvisiblespace" does, so I used \Vtextvisiblespace instead.
% https://tex.stackexchange.com/questions/50804/explicit-space-character
\newcommand\Vtextvisiblespace[1][.8em]{%
	\mbox{\kern.07em\vrule height.3ex}%
	\vbox{\hrule width#1}%
	\hbox{\vrule height.3ex}}

% Also found this on Stackoverflow for making tabs.
% https://tex.stackexchange.com/questions/198432/using-the-tab-command
\newcommand\tab[1][1cm]{\hspace*{#1}}

\begin{document}

\noindent
Jason Downing \\
Email: jason\_downing@student.uml.edu \\
Foundations of Computer Science \\
Homework \#4 - Chapter 3 \\
11/20/2016 \\

%*********************************************************************************
% Extra Credit Documentation!
%*********************************************************************************
**********************************************************************************
I was unable to complete a large portion of 3.8, including the state diagrams
and the configurations for those state diagrams. $:($ \\
To make up for these missing problems I decided to do as much \textbf{Extra Credit}
as I could. As a result, I did the following \textbf{Extra Credit} problems: \\

1. 3.1 B \\
2. 3.2 A \\
3. 3.4 \\
4. 3.5 A \\
5. 3.5 B \\
6. 3.5 C \\
7. 3.5 D \\
8. 3.6 \\ 
9. 3.7 \\
10. 3.8 A \\ 
11. 3.10 \\
12. 3.11 \\

I figured that I missed roughly 12 problems, so I should do up to 12 problems
worth of Extra Credit. I hope that this is enough extra work to counter the
missing problems that I was unable to solve before the deadline. \\

********************************************************************************** \\

%*********************************************************************************
% 3.1 a, b, c + 00000000
\pagebreak
\noindent
\textbf{3.1} This exercise concerns $TM M_2$, whose description and state diagram \\
appear in Example 3.7. In each of the parts, give the sequence of \\
configurations that $M_2$ enters when started on the indicated input string. \\

\textbf{a.} 0. \\

$q_1 0$ is the starting state. \\
Running the input 0 on the machine $M_2$, we get the following configuration: \\

$q_1 0$ \\
$\Vtextvisiblespace q_2 \Vtextvisiblespace$ \\
$\Vtextvisiblespace  \Vtextvisiblespace q_{accept}$  \\
$M_2$ enters into the $q_{accept}$ state, and as a result the input is accepted. \\

\textbf{b.} 00. (\textbf{THIS IS EXTRA CREDIT})

$q_1 00$ is the starting state. \\
Running the input 00 on the machine $M_2$, we get the following configuration: \\

$                            q_1 00$ \\
$ \Vtextvisiblespace         q_2  0$ \\
$ \Vtextvisiblespace x       q_3  \Vtextvisiblespace $ \\
$ \Vtextvisiblespace         q_5  x  \Vtextvisiblespace$ \\
$                            q_5 \Vtextvisiblespace x  \Vtextvisiblespace$ \\
$ \Vtextvisiblespace         q_2  x  \Vtextvisiblespace$ \\
$ \Vtextvisiblespace x       q_2 \Vtextvisiblespace $ \\
$ \Vtextvisiblespace x  \Vtextvisiblespace q_{accept} $ \\

\pagebreak
\textbf{c.} 000. \\
$q_1 000$ is the starting state. \\
Running the input 000 on the machine $M_2$, we get the following configuration: \\

$q_1 000$ \\
$\Vtextvisiblespace q_2 00$ \\
$\Vtextvisiblespace x q_3 0$ \\
$\Vtextvisiblespace x0 q_4 $ \\
$\Vtextvisiblespace x0 q_{reject} $ \\
$M_2$ enters into the $q_{reject}$ state, and as a result the input is rejected. \\

\textbf{d.} 000000. \\
$q_1 000000$ is the starting state. \\
Running the input 000000 on the machine $M_2$, we get the following config: \\

$q_1 000000$ \\
$\Vtextvisiblespace       q_2 00000 $  \\
$\Vtextvisiblespace x     q_3 0000 $ \\
$\Vtextvisiblespace x0    q_4 000 $ \\
$\Vtextvisiblespace x0x   q_3 00 $ \\
$\Vtextvisiblespace x0x0  q_4 0 $ \\
$\Vtextvisiblespace x0x0x q_3 \Vtextvisiblespace $ \\
$\Vtextvisiblespace x0x0  q_5 x \Vtextvisiblespace $ \\
$\Vtextvisiblespace x0x   q_5 0x \Vtextvisiblespace $ \\
$\Vtextvisiblespace x0    q_5 x0x \Vtextvisiblespace $ \\
$\Vtextvisiblespace x     q_5 0x0x \Vtextvisiblespace $ \\
$\Vtextvisiblespace       q_5 x0x0x \Vtextvisiblespace $ \\
$q_5 \Vtextvisiblespace x0x0x \Vtextvisiblespace $ \\
$\Vtextvisiblespace       q_2 x0x0x \Vtextvisiblespace $ \\
$\Vtextvisiblespace x     q_2 0x0x \Vtextvisiblespace $ \\
$\Vtextvisiblespace xx    q_3 x0x \Vtextvisiblespace $ \\
$\Vtextvisiblespace xxx   q_3 0x \Vtextvisiblespace $ \\
$\Vtextvisiblespace xxx0  q_4 x \Vtextvisiblespace $ \\
$\Vtextvisiblespace xxx0x q_4  \Vtextvisiblespace $ \\
$\Vtextvisiblespace xxx0x \Vtextvisiblespace q_{reject} $ \\
$M_2$ enters into the $q_{reject}$ state, and as a result the input is rejected. \\

\pagebreak
\textbf{Plus 00000000.} \\

$q_1 00000000$ is the starting state. \\
Running the input 000000 on the machine $M_2$, we get the following configuration: \\

$q_1 00000000$ \\
$\Vtextvisiblespace           q_2 0000000 $  \\ 
$\Vtextvisiblespace x         q_3 000000  $  \\ 
$\Vtextvisiblespace x0        q_4 00000   $  \\ 
$\Vtextvisiblespace x0x       q_3 0000    $  \\ 
$\Vtextvisiblespace x0x0      q_4 000     $  \\ 
$\Vtextvisiblespace x0x0x     q_3 00      $  \\ 
$\Vtextvisiblespace x0x0x0    q_4 0       $  \\ 
$\Vtextvisiblespace x0x0x0x   q_3         \Vtextvisiblespace  $  \\ 
$\Vtextvisiblespace x0x0x0    q_5 x       \Vtextvisiblespace  $  \\ 
$\Vtextvisiblespace x0x0x     q_5 0x      \Vtextvisiblespace  $  \\ 
$\Vtextvisiblespace x0x0      q_5 x0x     \Vtextvisiblespace  $  \\ 
$\Vtextvisiblespace x0x       q_5 0x0x    \Vtextvisiblespace  $  \\ 
$\Vtextvisiblespace x0        q_5 x0x0x   \Vtextvisiblespace  $  \\ 
$\Vtextvisiblespace x         q_5 0x0x0x  \Vtextvisiblespace  $  \\ 
$\Vtextvisiblespace           q_5 x0x0x0x \Vtextvisiblespace  $  \\ 
$q_5 \Vtextvisiblespace           x0x0x0x \Vtextvisiblespace  $  \\ 
$\Vtextvisiblespace           q_2 x0x0x0x \Vtextvisiblespace  $  \\
$\Vtextvisiblespace x         q_2 0x0x0x  \Vtextvisiblespace  $  \\
$\Vtextvisiblespace xx        q_3 x0x0x   \Vtextvisiblespace  $  \\
$\Vtextvisiblespace xxx       q_3 0x0x    \Vtextvisiblespace  $  \\
$\Vtextvisiblespace xxx0      q_4 x0x     \Vtextvisiblespace  $  \\
$\Vtextvisiblespace xxx0x     q_4 0x      \Vtextvisiblespace  $  \\
$\Vtextvisiblespace xxx0xx    q_3 x       \Vtextvisiblespace  $  \\
$\Vtextvisiblespace xxx0xxx   q_3         \Vtextvisiblespace  $  \\
$\Vtextvisiblespace xxx0xx    q_5 x       \Vtextvisiblespace  $  \\
$\Vtextvisiblespace xxx0x     q_5 xx      \Vtextvisiblespace  $  \\
$\Vtextvisiblespace xxx0      q_5 xxx     \Vtextvisiblespace  $  \\
$\Vtextvisiblespace xxx       q_5 xxx0    \Vtextvisiblespace  $  \\
$\Vtextvisiblespace xx        q_5 xxx0x   \Vtextvisiblespace  $  \\
$\Vtextvisiblespace x         q_5 xxx0xx  \Vtextvisiblespace  $  \\
$\Vtextvisiblespace           q_5 xxx0xxx \Vtextvisiblespace  $  \\
$q_5 \Vtextvisiblespace           xxx0xxx \Vtextvisiblespace  $  \\
$\Vtextvisiblespace           q_2 xxx0xxx \Vtextvisiblespace  $  \\
$\Vtextvisiblespace x         q_2 xx0xxx  \Vtextvisiblespace  $  \\
$\Vtextvisiblespace xx        q_2 x0xxx   \Vtextvisiblespace  $  \\
$\Vtextvisiblespace xxx       q_2 0xxx    \Vtextvisiblespace  $  \\
$\Vtextvisiblespace xxxx      q_3 xxx     \Vtextvisiblespace  $  \\
$\Vtextvisiblespace xxxxx     q_3 xx      \Vtextvisiblespace  $  \\
$\Vtextvisiblespace xxxxxx    q_3 x       \Vtextvisiblespace  $  \\
$\Vtextvisiblespace xxxxxxx   q_3         \Vtextvisiblespace  $  \\
$\Vtextvisiblespace xxxxxx    q_5 x       \Vtextvisiblespace  $  \\
$\Vtextvisiblespace xxxxx     q_5 xx      \Vtextvisiblespace  $  \\
$\Vtextvisiblespace xxxx      q_5 xxx     \Vtextvisiblespace  $  \\
$\Vtextvisiblespace xxx       q_5 xxxx    \Vtextvisiblespace  $  \\
$\Vtextvisiblespace xx        q_5 xxxxx   \Vtextvisiblespace  $  \\
$\Vtextvisiblespace x         q_5 xxxxxx  \Vtextvisiblespace  $  \\
$\Vtextvisiblespace           q_5 xxxxxxx \Vtextvisiblespace  $  \\
$q_5 \Vtextvisiblespace           xxxxxxx \Vtextvisiblespace  $  \\
$\Vtextvisiblespace           q_2 xxxxxxx \Vtextvisiblespace  $  \\
$\Vtextvisiblespace x         q_2 xxxxxx  \Vtextvisiblespace  $  \\
$\Vtextvisiblespace xx        q_2 xxxxx   \Vtextvisiblespace  $  \\
$\Vtextvisiblespace xxx       q_2 xxxx    \Vtextvisiblespace  $  \\
$\Vtextvisiblespace xxxx      q_2 xxx     \Vtextvisiblespace  $  \\
$\Vtextvisiblespace xxxxx     q_2 xx      \Vtextvisiblespace  $  \\
$\Vtextvisiblespace xxxxxx    q_2 x       \Vtextvisiblespace  $  \\
$\Vtextvisiblespace xxxxxxx   q_2         \Vtextvisiblespace  $  \\
$\Vtextvisiblespace xxxxxxx   q_{accept}  \Vtextvisiblespace  $  \\
$M_2$ enters into the $q_{accept}$ state, and as a result the input is accepted. \\

%*********************************************************************************
% 3.2 b, c, d, e + 01100#01100, 01101#01100
\pagebreak

\textbf{3.2} This exercise concerns $TM M_1$, whose description and state diagram appear in 
Example 3.9. In each of the parts, give the sequence of configurations that $M_1$ 
enters when started on the indicated input string. \\

\textbf{a.} $11$ (\textbf{THIS IS EXTRA CREDIT}) \\

Input is $11$. Starting state is $q_1 11$. Config is: \\

$      q_1 11$ \\
$ \Vtextvisiblespace q_3 1 $ \\
$ \Vtextvisiblespace 1 q_3 \Vtextvisiblespace $ \\
$ \Vtextvisiblespace 1 \Vtextvisiblespace q_{reject} $ \\
$M_1$ enters into the $q_{reject}$ state, and as a result the input is rejected. \\

The reason for this is because there is no path for the machine $M_1$ to take at
$q_3$ that involves a 1. It cannot go to $q_4$ without having a 0, and it cannot
go to $q_5$ without having a $\Vtextvisiblespace$ before the $q_3$. \\

\textbf{b.} $1\#1$. \\

Input is $1\#1$. Starting state is $q_1 1\#1$. \\
Running $1\#1$ on the machine $M_1$ results in the following configuration: \\

$       q_1 1 \# 1  $ \\
$x      q_3 \# 1    $ \\
$x \#   q_5 1       $ \\
$x \#   q_6 x       $ \\
$       q_7 x \# x  $ \\
$x      q_1 \# x    $ \\
$x \#   q_8 x       $ \\
$x \# x q_9 \Vtextvisiblespace $ \\
$x \# x \Vtextvisiblespace q_{accept} $ \\
$M_1$ enters into the $q_{accept}$ state, and as a result the input is accepted. \\

\pagebreak
\textbf{c.} $1\#\#1$ \\

Input is $1\#\#1$. Starting state is $q_1 1\#\#1$. \\
Running $1\#\#1$ on the machine $M_1$ results in the following configuration: \\

$q_1 1 \# \# 1$ \\
$x q_3 \# \# 1$ \\
$x \# q_5 \# 1$ \textit{(At this point $q_5$ does not read the $\#$, so it enters the reject state)} \\
$x \# \# q_{reject} 1 $ \\
$M_1$ enters into the $q_{reject}$ state, and as a result the input is rejected. \\

\textbf{d.} $10\#11$. \\

Input is $10\#11$. Starting state is $q_1 10\#11$. \\
Running $10\#11$ on the machine $M_1$ results in the following configuration: \\

$        q_1 10 \# 11  $ \\
$x       q_3 0 \# 11   $ \\
$x0      q_3 \# 11     $ \\
$x0 \#   q_5 11        $ \\
$x0      q_6 \# x 1    $ \\
$x       q_7 0 \# x 1    $ \\
$        q_7 x 0 \# x 1  $ \\
$x       q_1 0 \# x 1    $ \\
$xx      q_2 \# x 1      $ \\
$xx \#   q_4 x 1         $ \\
$xx \# x q_4 1 $ \textit{(At this point $q_4$ does not read the $1$, so it enters the reject state)} \\
$xx \# x 1 q_{reject}    $ \\
$M_1$ enters into the $q_{reject}$ state, and as a result the input is rejected. \\

\pagebreak
\textbf{e.} $10\#10$. \\

Input is $10\#10$. Starting state is $q_1 10\#10$. \\
Running $10\#10$ on the machine $M_1$ results in the following configuration: \\

$            q_1 10 \# 10   $ \\
$x           q_3 0  \# 10   $ \\
$x0          q_3    \# 10   $ \\
$x0 \#       q_5       10   $ \\
$x0          q_6    \# x0   $ \\
$x           q_7  0 \# x0   $ \\
$            q_7 x0 \# x0   $ \\
$x           q_1  0 \# x0   $ \\
$xx          q_2    \# x0   $ \\
$xx \#       q_4       x0   $ \\
$xx \# x     q_4        0   $ \\
$xx \#       q_6       xx   $ \\
$xx          q_6    \# xx   $ \\
$x           q_7  x \# xx   $ \\
$xx          q_1    \# xx   $ \\
$xx \#       q_8       xx   $ \\
Now right shift $q_8$ until all x's have been read \\
$xx \# xx    q_8  \Vtextvisiblespace   $ \\
$xx \# xx         \Vtextvisiblespace q_{accept} \Vtextvisiblespace $ \\
$M_1$ enters into the $q_{accept}$ state, and as a result the input is accepted. \\

% Plus stuff here!
\pagebreak
\textbf{Plus:} $01100\#01100$ \\

Input is $01100\#01100$. Starting state is $q_1 01100\#0110$. \\
Running $01100\#01100$ on the machine $M_1$ results in the following config: \\

$                     q_1 01100 \# 01100$ \\
$x                    q_2  1100 \# 01100$ \\
Right shift (0,1) until we hit a $\#$.  \\
$x1100                q_2       \# 01100$ \\
$x1100 \#             q_4          01100$ \\
$x1100 \# x           q_6           1100$ \\
$x1100 \#             q_6          x1100$ \\
$x1100                q_7       \# x1100$ \\
Left shift (0,1) until we hit a x.      \\
$x                    q_7  1100 \# x1100$ \\
$x1                   q_1   100 \# x1100$ \\
$x1x                  q_3    00 \# x1100$ \\
Right shift (0,1) until we hit a $\#$.  \\
$x1x00                q_3       \# x1100$ \\
$x1x00 \#             q_5          x1100$ \\
$x1x00 \# x           q_5           1100$ \\
$x1x00 \# xx          q_6            100$ \\
Left shift (0,1,x) until we hit a $\#$.  \\
$x1x00 \#             q_6          xx100$ \\
$x1x00                q_7       \# xx100$ \\
Left shift (0,1) until we hit a x.        \\
$x1x                  q_7    00 \# xx100$ \\
$x1                   q_1   x00 \# xx100$ \\
$xxx                  q_3    00 \# xx100$ \\
Right shift (0,1) until we hit a $\#$.  \\
$xxx00                q_3       \# xx100$ \\
$xxx00 \#             q_5          xx100$ \\
Right shift (x) until we hit a 1.
$xxx00 \# xx          q_5            100$ \\
$xxx00 \# x           q_6           xx00$ \\
$xxx00 \#             q_6          xxx00$ \\
$xxx00                q_7       \# xxx00$ \\
Left shift (0,1) until we hit a x.        \\
$xxx                  q_7    00 \# xxx00$ \\
$xxx0                 q_1     0 \# xxx00$ \\
$xxxx0                q_2       \# xxx00$ \\
$xxxx0 \#             q_4          xxx00$ \\
$xxxx0 \# x           q_4           xx00$ \\
Right shift (x) until we hit a 0.         \\
$xxxx0 \# xxx         q_4             00$ \\
$xxxx0 \# xx          q_6            xx0$ \\
Left shift (0,1,x) until we hit a $\#$.  \\
$xxxx0 \#             q_6            xxxx0$ \\
$xxxx0                q_7        \#  xxxx0$ \\
$xxxx                 q_7      0 \#  xxxx0$ \\
$xxxx0                q_1        \#  xxxx0$ \\
$xxxxx \#             q_2            xxxx0$ \\
$xxxxx \# x           q_4             xxx0$ \\
Right shift until we hit a 0. \\
$xxxxx \# xxxx        q_4                0$ \\
$xxxxx \# xxx         q_6               xx$ \\
Left shift (0,1,x) until we hit a $\#$. \\
$xxxxx \#             q_6            xxxxx$ \\
$xxxxx                q_7        \#  xxxxx$ \\
$xxxxx \#             q_1            xxxxx$ \\
$xxxxx \# x           q_8             xxxx$ \\
Right shift all the x's. \\
$xxxxx \# xxxxx       q_8  \Vtextvisiblespace    $ \\
$xxxxx \# xxxxx  \Vtextvisiblespace  q_{accept} \Vtextvisiblespace $ \\
$M_1$ enters into the $q_{accept}$ state, and as a result the input is accepted. \\

\pagebreak
\textbf{Plus:} $01101\#01100$ \\

Input is $01101\#01100$. Starting state is $q_1 01101\#01100$. \\
Running $01101\#01100$ on the machine $M_1$ results in the following config: \\

$                   q_1 01101 \# 01100$ \\
$x                  q_2  1101 \# 01100$ \\
Right shift (0,1) until we hit a $\#$   \\
$x1101              q_2       \# 01100$ \\
$x1101 \#           q_4          01100$ \\
$x1101              q_6       \# x1100$ \\
$x110               q_7     1 \# x1100$ \\
Left shift (0,1) until we hit a x       \\
$x                  q_7  1101 \# x1100$ \\
$x1                 q_1   101 \# x1100$ \\
$xx1                q_3    01 \# x1100$ \\
Right shift (0,1) until we hit a $\#$   \\
$xx101              q_3       \# x1100$ \\
$xx101 \#           q_5          x1100$ \\
$xx101 \# x         q_5           1100$ \\
$xx101 \#           q_6          xx100$ \\
$xx101              q_7       \# xx100$ \\
Left shift (0,1) until we hit a x       \\
$xx                 q_7   101 \# xx100$ \\
$xx1                q_1    01 \# xx100$ \\
$xxx0               q_3     1 \# xx100$ \\
$xxx01              q_3       \# xx100$ \\
$xxx01 \#           q_5          xx100$ \\
Right shift (x) until we hit a 1.       \\
$xxx01 \# xx        q_5            100$ \\
$xxx01 \# x         q_6           xx00$ \\
$xxx01 \#           q_6          xxx00$ \\
$xxx01              q_7       \# xxx00$ \\
Left shift (0,1) until we hit a x       \\
$xxx                q_7    01 \# xxx00$ \\
$xxx0               q_1     1 \# xxx00$ \\
$xxxx1              q_2       \# xxx00$ \\
$xxxx1 \#           q_4          xxx00$ \\
Right shift (x) until we hit a 0.       \\
$xxxx1 \# xxx       q_4             00$ \\
$xxxx1 \# xxxx      q_6              0$ \\
Left shift (0, 1, x) until we hit  $\#$ \\
$xxxx1 \#           q_6          xxxx0$ \\
$xxxx1              q_7      \#  xxxx0$ \\
$xxxx               q_7    1 \#  xxxx0$ \\
$xxxx1              q_1      \#  xxxx0$ \\
$xxxxx  \#          q_3          xxxx0$ \\
$xxxxx              q_5      \#  xxxx0$ \\
Right shift (x) until we hit a 1.
$xxxxx \#  xxxx     q_5              0$ \\
At this point the machine will go into the reject state. \\
The reason for this is that it cannot read a 1, or an x. \\
$xxxxx \#  xxxx0    q_{reject}        $ \\
$M_1$ enters into the $q_{reject}$ state, and as a result the input is rejected. \\

The reason for this failure is that at $q_5$, the machine expects a 1 but it gets \\
a 0. There are no paths for 0 at $q_5$, and as a result the machine will go into \\
the reject state as it is unable to continue. \\

%*********************************************************************************
% 3.4 here

\pagebreak
\textbf{3.4} (\textbf{THIS IS EXTRA CREDIT}) \\
Give a formal definition of an enumerator. Consider it to be a type of two-tape
Turing machine that uses its second tape as the printer. Include a definition
of the enumerated language. \\

We can define an enumerator as a 7-tuple:
($Q$, $\sum$, $\tau$, $\delta$, $q_0$, $q_{print}$, $q_{accept}$) \\

Where $Q$, $\sum$, and $\tau$ are finite sets and:

1. $Q$ is the set of states \\
2. $\tau$ is the work tape alphabet \\
3. $\sum$ is the output tape alphabet \\
4. $\delta:$ $Q \times \tau \rightarrow Q \times \tau \times \{L, R\} \times \sum_{\epsilon}$
is the transition function \\
5. $q_0 \epsilon Q$ is the starting state \\
6. $q_{print} \epsilon Q$ is the printing state \\
7. $q_{accept} \epsilon Q$ is the rejecting state, which is where $q_{print} \ne q_{reject}$ \\

The computation of this enumerator E is defined as an ordinary Turing Machine (TM),
except for a few points. It has two tapes, a print tape and a work tape, which
both begin blank initially. During each step, the TM can write a symbol from $\sum$
onto the output tape, or it can write nothing at all, which is determined by $\delta$.
If $\delta (q, a) = (r, b, L, c)$, it means that during state $q$ while reading a,
the enumerator E enters into state R, and it writes B back onto the work tape. It
then moves the head of the working tape left or right depending on whether L was
previously R, and it then writes C onto the output tape. If $C \ne \epsilon$, then
the head of the output tape moves to the right. \\

Whenever we enter the state $q_{print}$, then the output tape is reset to blank,
and the head returns to the left side. The machine will halt when $q_{accept}$ is
entered. $L(E) = \{w \epsilon \sum^* | w \}$ will be on the work tape if $q_{print}$
is entered. \\

%*********************************************************************************
% 3.5 a, b, c an d here

\pagebreak
\textbf{3.5} (\textbf{THIS IS EXTRA CREDIT}) \\
Examine the formal definition of a Turing machine to answer the following questions,
and explain your reasoning. \\

\textbf{a.} Can a Turing machine ever write the blank symbol $\Vtextvisiblespace$ on its tape? \\

Yes, a turing machine can write the blank symbol $\Vtextvisiblespace$ onto its tape.
This is because according to the definition of a Turing Machine, a TM may write any
characters in $\tau$ onto its tape. $\tau$ is the tape alphabet, and according to the
definition of a TM, $\Vtextvisiblespace \in \tau $. \\

\textbf{b.} Can the tape alphabet $\tau$ be the same as the input alphabet $\sum$? \\

No, this is not possible because $\sum$ never contains the blank symbol $\Vtextvisiblespace$.
However, $\tau$ always contains the blank symbol $\Vtextvisiblespace$, and as a result,
they can never be equal. \\

\textbf{c.} Can a Turing machine’s head \textit{ever} be in the same location in two successive 
steps? \\

Yes, a TM can have its head in the same location during two successive steps. This is because
if the TM tries to move its head of the left hand side of the tape, it will remain in the same
location as the previous step. \\

\textbf{d.} Can a Turing machine contain just a single state? \\

No, a TM cannot contain just a single state. Any TM by definition must contain at least
two distinct states, $q_{accept}$ and $q_{reject}$. As a result, a TM must contain
at a minimum two states to be considered a TM. \\

%*********************************************************************************
% 3.6 here
\pagebreak
\textbf{3.6} (\textbf{THIS IS EXTRA CREDIT}) \\

In Theorem 3.21, we showed that a language is Turing-recognizable iff some 
enumerator enumerates it. Why didn’t we use the following simpler algorithm for the
forward direction of the proof? As before, $s_1$, $s_2$, ... is a list of all strings in $\sum^*$. \\

$E = $"Ignore the input. \\
\tab 1. Repeat the following for $i = 1, 2, 3, ...$ \\
\tab 2. Run $M$ on $s_1$ \\
\tab 3. If it accepts, print ouf $s_1$." \\

The reason we don't use a simpler algorithm during this proof is that in part 2 
of the algorithm (Run $M$ on $s_1$), if $M$ loops on the input $s_i$, $E$ will not 
check any inputs after $s_i$. As a result, if that were to occur, $E$ could fail to
enumerate L(M) as it is required to. \\

%*********************************************************************************
% 3.7 here
\textbf{3.7} (\textbf{THIS IS EXTRA CREDIT}) \\

Explain why the following is not a description of a legitimate Turing machine.

$M_{bad} =$ "On input ($p$), a polynomial over variables $x_1$, ..., $x_k$: \\
\tab 1. Try all possible settings of $x_1$, ..., $x_k$ to integer values. \\
\tab 2. Evaluate $p$ on all of these settings. \\
\tab 3. If any of these settings evaluates to 0, $accept$ or otherwise $reject$. \\

This is not a description of a legit Turing Machine because the variables listed
have an infinite amount of possible settings. A TM would as a result require an
infinite amount of time to try all possible settings. This is not allowed in a
valid TM as we require that each step in a Turing Machine be completed in a finite
amount of steps. Requiring an infinite amount of steps as a result makes it not
a legitimate Turing Machine. \\

%*********************************************************************************
% 3.8  b, c + other stuff
\pagebreak
\textbf{3.8} Give implementation-level descriptions of Turing machines that decide the following
languages over the alphabet $\{0,1\}$. \\

\textbf{a.} $\{$w $\mid$ w contains an equal number of 0's and 1's. \\
(\textbf{THIS IS EXTRA CREDIT})  \\

For input string $w$, we would do the following implementation: \\
\textbf{1.} Scan the tape, and mark the first 0 which has not been marked
yet. Now if there are no unmarked 0's, we go to stage $\# 4$. Otherwise,
we move the head back to the front of the tape. \\

\textbf{2.} We now scan the tape, and mark the first 1 which has no been
marked yet. If we find no unmarked 1's, we enter the $reject$ state. \\

\textbf{3.} Now move the head back to the front of the tape, and repeat
step $\# 1$. \\

\textbf{4.} Now move the head to the front of the tape. We scan the tape
to see if we can find any unmarked 1's that are remaining. If we find no
unmarked 1's, we enter the $accept$ state. Otherwise, we enter the $reject$
state. \\

\textbf{b.} $\{$w $\mid$ w contains twice as many 0s as 1s$\}$ \\

For input string $w$, we would do the following implementation: \\
\textbf{1.} We first scan the tape and mark the first 0 that has not been marked yet.
If we find no unmarked 0's, we then continue to $\#4$. \\

\textbf{2.} We move onto mark the next unmarked 0. If we do not find any on the tape,
we enter the reject state. Otherwise, we move back to the front of the tape. \\

\textbf{3.} We scan the tape and mark the first 1 which has not been marked yet. If there
is no unmarked 1, we enter the reject state. \\

\textbf{4.} We now move the head back to the front of the tape, and repeat $\#1$. \\

\textbf{5.} We move the head of the tape back to the front of the tape, and we then scan
the tape to see if we can find any unmarked 1's. If there are none, we enter the
accept state. Otherwise we enter the reject state. \\

\textbf{c.} $\{$w $\mid$ w does not contain twice as many 0s as 1s$\}$ \\

For input string $w$, we would do the following implementation: \\
\textbf{1.} We first scan the tape and mark the first 0 which has not yet been marked.
If we find no unmarked 0, we go to $\#4$. \\

\textbf{2.} We continue moving and mark the next unmarked 0. If we do not find any on
the tape, then we enter the accept state. Otherwise, we move the head of the tape back
to the front and we continue. \\

\textbf{3.} We scan the tape and mark the first 1 which has not yet been marked. If 
there are no unmarked 1's, we enter the accept state. \\

\textbf{4.} We move the head back to the front of the tape, and we repeat $\#1$. \\

\textbf{5.} We move the head back to the front of the tape, and we scan the tape
to see if there are any unmarked 1's. If there are no unmarked 1's, we enter the 
reject state. Otherwise, we enter the accept state. \\

%*********************************************************************************
% 3.10 here

\pagebreak
3.10 (\textbf{THIS IS EXTRA CREDIT}) \\

Say that a write-once Turing machine is a single-tape TM that can alter each tape
square at most once (including the input portion of the tape). Show that this variant
Turing machine model is equivalent to the ordinary Turing machine model. 
(Hint: As a first step, consider the case whereby the Turing machine may alter each
tape square at most twice. Use lots of tape.) \\

We can simulate an ordinary Turing Machine (TM) by using a write-twice TM. The
write-twice TM can simulate a single step of the original TM by copying its entire
tape over to a new section of the tape to the right of the currently used section.
The copy procedure would be as follows: \\

$\bullet$ Copy character, by character, marking a character as it is copied. \\
$\bullet$ Alter the tape square twice, once to write a character for the first time, and
a second time to mark that it has been copied. \\

The position of the original TM's tape head is marked on the tape, and when it
copies the positions around the marked position, the tape contents are updated
according to the rules of the TM. \\

To complete the simulation with a write-once TM, we would operate the same as before,
except that we would make sure that each cell of the tape is represented by two cells.
The first cell would contain the original TM's tape, and the second would be used to
mark each character as it is copied over during the copying procedure. The input would
not be presented to the machine as there are two cells per symbol, so as a result the
first time that the tape is copied the marks are put directly over the input symbols. \\

%*********************************************************************************
% 3.11 here

\pagebreak
3.11 (\textbf{THIS IS EXTRA CREDIT}) \\

A \textbf{\textit{Turing machine with doubly infinite tape}} is similar to an ordinary 
Turing machine, but its tape is infinite to the left as well as to the right. 
The tape is initially filled with blanks except for the portion that contains the input. 
Computation is defined as usual except that the head never encounters an end to the tape 
as it moves leftward. Show that this type of Turing machine recognizes the class of
Turing-recognizable languages. \\

A Turing Machine (TM) with \textbf{\textit{"doubly infinite tape"}} can easily
simulate a normal TM. It will need to mark the left side of the input's end, so
that it can prevent the head from moving off the end of the tape. \\

In order for a normal TM to simulate a \textbf{\textit{"doubly infinite tape"}} TM,
we will need to show how to simulate the TM with a 2 tape TM. This has already been
shown to be equal in power to a normal TM. The first tape of this 2 tape TM will be
written with the input string, and the second tape will be left blank initially.
We will cut the tape of the doubly infinite tape TM into two separate parts, which
will be at the starting position of the input string. The input string portion and
the blank spaces to its right will appear on the first tape of the 2 tape TM. The
portion to the left of the input string will appear on the second tape, but in
the reverse order. \\

%*********************************************************************************
% Extra Credit Documentation!
%*********************************************************************************
\pagebreak
**********************************************************************************
Everything else from here on I could not figure out how to do. $:($
To make up for these missing problems I decided to do as much \textbf{Extra Credit}
as I could. As a result, I did the following \textbf{Extra Credit} problems: \\

 1. 3.1 B \\
 2. 3.2 A \\
 3. 3.4 \\
 4. 3.5 A \\
 5. 3.5 B \\
 6. 3.5 C \\
 7. 3.5 D \\
 8. 3.6 \\ 
 9. 3.7 \\
10. 3.8 A \\ 
11. 3.10 \\
12. 3.11 \\

I figured that I missed roughly 12 problems, so I should do up to 12 problems
worth of Extra Credit. I hope that this is enough extra work to counter the
missing problems that I was unable to solve before the deadline. \\

********************************************************************************** \\


%*********************************************************************************
% I couldn't do this other stuff. Sorry. ¯\_(ツ)_/¯
\pagebreak
Plus: Draw the state diagram for Turning Machines 3.8b and 3.8c \\
3.8b \\

3.8c \\

For these machines draw the configurations for \\
3.8b: \\
010100. \\

010101. \\

3.8c: \\
000111. \\

000110. \\

%*********************************************************************************
% Modify machine M2 to recognize odd number of 0s and draw the state diagram. 
% Draw the configuration for 000 and 0000
\pagebreak

Modify machine M2 to recognize odd number of 0s and draw the state \\
diagram. \\

State diagram: \\

Draw the configuration for \\
000. \\

0000. \\

%*********************************************************************************
% Modify machine M2 to recognize even number of 0s and draw the state diagram. 
% Draw the configuration for 000 and 0000
\pagebreak

Modify machine M2 to recognize even number of 0s and draw the state \\
diagram. \\

State diagram: \\

Draw the configuration for \\
000. \\

0000. \\

\end{document}
