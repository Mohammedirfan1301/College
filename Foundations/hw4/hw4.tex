\documentclass[12pt]{article}
\setlength\parindent{0pt}
\usepackage{amsfonts}
\usepackage{comment}
\usepackage{tikz}
\usepackage{forest}
\usetikzlibrary{arrows,automata}

% Found this on Stackoverflow, seems to make a better explicit space than the default
% "\textvisiblespace" does, so I used \Vtextvisiblespace instead.
% https://tex.stackexchange.com/questions/50804/explicit-space-character
\newcommand\Vtextvisiblespace[1][.8em]{%
	\mbox{\kern.07em\vrule height.3ex}%
	\vbox{\hrule width#1}%
	\hbox{\vrule height.3ex}}

\begin{document}

\noindent
Jason Downing \\
Email: jason\_downing@student.uml.edu \\
Foundations of Computer Science \\
Homework \#4 - Chapter 3 \\
11/20/2016 \\

%*********************************************************************************
% 3.1 a, b, c + 00000000

\noindent
3.1 This exercise concerns $TM M_2$, whose description and state diagram \\
appear in Example 3.7. In each of the parts, give the sequence of \\
configurations that $M_2$ enters when started on the indicated input string. \\

a. 0. \\

$q_10$ is the starting state. \\
Running the input 0 on the machine $M_2$, we get the following configuration: \\

$q_1 0$ \\
$\Vtextvisiblespace q_2 \Vtextvisiblespace$ \\
$\Vtextvisiblespace  \Vtextvisiblespace q_{accept}$  \\
$M_2$ enters into the $q_{accept}$ state, and as a result the input is accepted. \\

c. 000. \\

$q_1 000$ is the starting state. \\
Running the input 000 on the machine $M_2$, we get the following configuration: \\

$q_1 000$ \\
$\Vtextvisiblespace q_2 00$ \\
$\Vtextvisiblespace x q_3 0$ \\
$\Vtextvisiblespace x0 q_4 $ \\
$\Vtextvisiblespace x0 q_{reject} $ \\
$M_2$ enters into the $q_{reject}$ state, and as a result the input is rejected. \\

\pagebreak
d. 000000. \\

$q_1000000$ is the starting state. \\
Running the input 000000 on the machine $M_2$, we get the following configuration: \\

$q_1 000000$ \\
$\Vtextvisiblespace       q_2 00000 $  \\
$\Vtextvisiblespace x     q_3 0000 $ \\
$\Vtextvisiblespace x0    q_4 000 $ \\
$\Vtextvisiblespace x0x   q_3 00 $ \\
$\Vtextvisiblespace x0x0  q_4 0 $ \\
$\Vtextvisiblespace x0x0x q_3 \Vtextvisiblespace $ \\
$\Vtextvisiblespace x0x0  q_5 x \Vtextvisiblespace $ \\
$\Vtextvisiblespace x0x   q_5 0x \Vtextvisiblespace $ \\
$\Vtextvisiblespace x0    q_5 x0x \Vtextvisiblespace $ \\
$\Vtextvisiblespace x     q_5 0x0x \Vtextvisiblespace $ \\
$\Vtextvisiblespace       q_5 x0x0x \Vtextvisiblespace $ \\
$q_5 \Vtextvisiblespace x0x0x \Vtextvisiblespace $ \\
$\Vtextvisiblespace       q_2 x0x0x \Vtextvisiblespace $ \\
$\Vtextvisiblespace x     q_2 0x0x \Vtextvisiblespace $ \\
$\Vtextvisiblespace xx    q_3 x0x \Vtextvisiblespace $ \\
$\Vtextvisiblespace xxx   q_3 0x \Vtextvisiblespace $ \\
$\Vtextvisiblespace xxx0  q_4 x \Vtextvisiblespace $ \\
$\Vtextvisiblespace xxx0x q_4  \Vtextvisiblespace $ \\
$\Vtextvisiblespace xxx0x \Vtextvisiblespace q_{reject} $ \\
$M_2$ enters into the $q_{reject}$ state, and as a result the input is rejected. \\

\pagebreak
Plus 00000000. \\

$q_100000000$ is the starting state. \\
Running the input 000000 on the machine $M_2$, we get the following configuration: \\

$q_1 00000000$ \\
$\Vtextvisiblespace           q_2 0000000 $  \\ 
$\Vtextvisiblespace x         q_3 000000  $  \\ 
$\Vtextvisiblespace x0        q_4 00000   $  \\ 
$\Vtextvisiblespace x0x       q_3 0000    $  \\ 
$\Vtextvisiblespace x0x0      q_4 000     $  \\ 
$\Vtextvisiblespace x0x0x     q_3 00      $  \\ 
$\Vtextvisiblespace x0x0x0    q_4 0       $  \\ 
$\Vtextvisiblespace x0x0x0x   q_3         \Vtextvisiblespace  $  \\ 
$\Vtextvisiblespace x0x0x0    q_5 x       \Vtextvisiblespace  $  \\ 
$\Vtextvisiblespace x0x0x     q_5 0x      \Vtextvisiblespace  $  \\ 
$\Vtextvisiblespace x0x0      q_5 x0x     \Vtextvisiblespace  $  \\ 
$\Vtextvisiblespace x0x       q_5 0x0x    \Vtextvisiblespace  $  \\ 
$\Vtextvisiblespace x0        q_5 x0x0x   \Vtextvisiblespace  $  \\ 
$\Vtextvisiblespace x         q_5 0x0x0x  \Vtextvisiblespace  $  \\ 
$\Vtextvisiblespace           q_5 x0x0x0x \Vtextvisiblespace  $  \\ 
$q_5 \Vtextvisiblespace           x0x0x0x \Vtextvisiblespace  $  \\ 
$\Vtextvisiblespace           q_2 x0x0x0x \Vtextvisiblespace  $  \\
$\Vtextvisiblespace x         q_2 0x0x0x  \Vtextvisiblespace  $  \\
$\Vtextvisiblespace xx        q_3 x0x0x   \Vtextvisiblespace  $  \\
$\Vtextvisiblespace xxx       q_3 0x0x    \Vtextvisiblespace  $  \\
$\Vtextvisiblespace xxx0      q_4 x0x     \Vtextvisiblespace  $  \\
$\Vtextvisiblespace xxx0x     q_4 0x      \Vtextvisiblespace  $  \\
$\Vtextvisiblespace xxx0xx    q_3 x       \Vtextvisiblespace  $  \\
$\Vtextvisiblespace xxx0xxx   q_3         \Vtextvisiblespace  $  \\
$\Vtextvisiblespace xxx0xx    q_5 x       \Vtextvisiblespace  $  \\
$\Vtextvisiblespace xxx0x     q_5 xx      \Vtextvisiblespace  $  \\
$\Vtextvisiblespace xxx0      q_5 xxx     \Vtextvisiblespace  $  \\
$\Vtextvisiblespace xxx       q_5 xxx0    \Vtextvisiblespace  $  \\
$\Vtextvisiblespace xx        q_5 xxx0x   \Vtextvisiblespace  $  \\
$\Vtextvisiblespace x         q_5 xxx0xx  \Vtextvisiblespace  $  \\
$\Vtextvisiblespace           q_5 xxx0xxx \Vtextvisiblespace  $  \\
$q_5 \Vtextvisiblespace           xxx0xxx \Vtextvisiblespace  $  \\
$\Vtextvisiblespace           q_2 xxx0xxx \Vtextvisiblespace  $  \\
$\Vtextvisiblespace x         q_2 xx0xxx  \Vtextvisiblespace  $  \\
$\Vtextvisiblespace xx        q_2 x0xxx   \Vtextvisiblespace  $  \\
$\Vtextvisiblespace xxx       q_2 0xxx    \Vtextvisiblespace  $  \\
$\Vtextvisiblespace xxxx      q_3 xxx     \Vtextvisiblespace  $  \\
$\Vtextvisiblespace xxxxx     q_3 xx      \Vtextvisiblespace  $  \\
$\Vtextvisiblespace xxxxxx    q_3 x       \Vtextvisiblespace  $  \\
$\Vtextvisiblespace xxxxxxx   q_3         \Vtextvisiblespace  $  \\
$\Vtextvisiblespace xxxxxx    q_5 x       \Vtextvisiblespace  $  \\
$\Vtextvisiblespace xxxxx     q_5 xx      \Vtextvisiblespace  $  \\
$\Vtextvisiblespace xxxx      q_5 xxx     \Vtextvisiblespace  $  \\
$\Vtextvisiblespace xxx       q_5 xxxx    \Vtextvisiblespace  $  \\
$\Vtextvisiblespace xx        q_5 xxxxx   \Vtextvisiblespace  $  \\
$\Vtextvisiblespace x         q_5 xxxxxx  \Vtextvisiblespace  $  \\
$\Vtextvisiblespace           q_5 xxxxxxx \Vtextvisiblespace  $  \\
$q_5 \Vtextvisiblespace           xxxxxxx \Vtextvisiblespace  $  \\
$\Vtextvisiblespace           q_2 xxxxxxx \Vtextvisiblespace  $  \\
$\Vtextvisiblespace x         q_2 xxxxxx  \Vtextvisiblespace  $  \\
$\Vtextvisiblespace xx        q_2 xxxxx   \Vtextvisiblespace  $  \\
$\Vtextvisiblespace xxx       q_2 xxxx    \Vtextvisiblespace  $  \\
$\Vtextvisiblespace xxxx      q_2 xxx     \Vtextvisiblespace  $  \\
$\Vtextvisiblespace xxxxx     q_2 xx      \Vtextvisiblespace  $  \\
$\Vtextvisiblespace xxxxxx    q_2 x       \Vtextvisiblespace  $  \\
$\Vtextvisiblespace xxxxxxx   q_2         \Vtextvisiblespace  $  \\
$\Vtextvisiblespace xxxxxxx   q_{accept}  \Vtextvisiblespace  $  \\
$M_2$ enters into the $q_{accept}$ state, and as a result the input is accepted. \\

%*********************************************************************************
% 3.2 b, c, d, e + 01100#01100, 01101#01100
\pagebreak

3.2 This exercise concerns $TM M_1$, whose description and state diagram appear in 
Example 3.9. In each of the parts, give the sequence of configurations that $M_1$ 
enters when started on the indicated input string. \\

b. $1\#1$. \\

Input is $1\#1$. Starting state is $q_1 1\#1$. \\
Running $1\#1$ on the machine $M_1$ results in the following configuration: \\

$       q_1 1 \# 1  $ \\
$x      q_3 \# 1    $ \\
$x \#   q_5 1       $ \\
$x \#   q_6 x       $ \\
$       q_7 x \# x  $ \\
$x      q_1 \# x    $ \\
$x \#   q_8 x       $ \\
$x \# x q_9 \Vtextvisiblespace $ \\
$x \# x \Vtextvisiblespace q_{accept} $ \\
$M_1$ enters into the $q_{accept}$ state, and as a result the input is accepted. \\

c. $1\#\#1$ \\

Input is $1\#\#1$. Starting state is $q_1 1\#\#1$. \\
Running $1\#\#1$ on the machine $M_1$ results in the following configuration: \\

$q_1 1 \# \# 1$ \\
$x q_3 \# \# 1$ \\
$x \# q_5 \# 1$ \textit{(At this point $q_5$ does not read the $\#$, so it enters the reject state)} \\
$x \# \# q_{reject} 1 $ \\
$M_1$ enters into the $q_{reject}$ state, and as a result the input is rejected. \\

\pagebreak
d. $10\#11$. \\

Input is $10\#11$. Starting state is $q_1 10\#11$. \\
Running $10\#11$ on the machine $M_1$ results in the following configuration: \\

$        q_1 10 \# 11  $ \\
$x       q_3 0 \# 11   $ \\
$x0      q_3 \# 11     $ \\
$x0 \#   q_5 11        $ \\
$x0      q_6 \# x 1    $ \\
$x       q_7 0 \# x 1    $ \\
$        q_7 x 0 \# x 1  $ \\
$x       q_1 0 \# x 1    $ \\
$xx      q_2 \# x 1      $ \\
$xx \#   q_4 x 1         $ \\
$xx \# x q_4 1 $ \textit{(At this point $q_4$ does not read the $1$, so it enters the reject state)} \\
$xx \# x 1 q_{reject}    $ \\
$M_1$ enters into the $q_{reject}$ state, and as a result the input is rejected. \\

\pagebreak
e. $10\#10$. \\

Input is $10\#10$. Starting state is $q_1 10\#10$. \\
Running $10\#10$ on the machine $M_1$ results in the following configuration: \\

$            q_1 10 \# 10   $ \\
$x           q_3 0  \# 10   $ \\
$x0          q_3    \# 10   $ \\
$x0 \#       q_5       10   $ \\
$x0          q_6    \# x0   $ \\
$x           q_7  0 \# x0   $ \\
$            q_7 x0 \# x0   $ \\
$x           q_1  0 \# x0   $ \\
$xx          q_2    \# x0   $ \\
$xx \#       q_4       x0   $ \\
$xx \# x     q_4        0   $ \\
$xx \#       q_6       xx   $ \\
$xx          q_6    \# xx   $ \\
$x           q_7  x \# xx   $ \\
$xx          q_1    \# xx   $ \\
$xx \#       q_8       xx   $ \\
Now right shift $q_8$ until all x's have been read \\
$xx \# xx    q_8  \Vtextvisiblespace   $ \\
$xx \# xx         \Vtextvisiblespace q_{accept} \Vtextvisiblespace $ \\
$M_1$ enters into the $q_{accept}$ state, and as a result the input is accepted. \\

% Plus stuff here!
\pagebreak
Also do this one: $01100\#01100$ \\

Input is $01100\#01100$. Starting state is $q_1 01100\#0110$. \\
Running $01100\#01100$ on the machine $M_1$ results in the following config: \\

$                     q_1 01100 \# 01100$ \\
$x                    q_2  1100 \# 01100$ \\
Right shift (0,1) until we hit a $\#$.  \\
$x1100                q_2       \# 01100$ \\
$x1100 \#             q_4          01100$ \\
$x1100 \# x           q_6           1100$ \\
$x1100 \#             q_6          x1100$ \\
$x1100                q_7       \# x1100$ \\
Left shift (0,1) until we hit a x.      \\
$x                    q_7  1100 \# x1100$ \\
$x1                   q_1   100 \# x1100$ \\
$x1x                  q_3    00 \# x1100$ \\
Right shift (0,1) until we hit a $\#$.  \\
$x1x00                q_3       \# x1100$ \\
$x1x00 \#             q_5          x1100$ \\
$x1x00 \# x           q_5           1100$ \\
$x1x00 \# xx          q_6            100$ \\
Left shift (0,1,x) until we hit a $\#$.  \\
$x1x00 \#             q_6          xx100$ \\
$x1x00                q_7       \# xx100$ \\
Left shift (0,1) until we hit a x.        \\
$x1x                  q_7    00 \# xx100$ \\
$x1                   q_1   x00 \# xx100$ \\
$xxx                  q_3    00 \# xx100$ \\
Right shift (0,1) until we hit a $\#$.  \\
$xxx00                q_3       \# xx100$ \\
$xxx00 \#             q_5          xx100$ \\
Right shift (x) until we hit a 1.
$xxx00 \# xx          q_5            100$ \\
$xxx00 \# x           q_6           xx00$ \\
$xxx00 \#             q_6          xxx00$ \\
$xxx00                q_7       \# xxx00$ \\
Left shift (0,1) until we hit a x.        \\
$xxx                  q_7    00 \# xxx00$ \\
$xxx0                 q_1     0 \# xxx00$ \\
$xxxx0                q_2       \# xxx00$ \\
$xxxx0 \#             q_4          xxx00$ \\
$xxxx0 \# x           q_4           xx00$ \\
Right shift (x) until we hit a 0.         \\
$xxxx0 \# xxx         q_4             00$ \\
$xxxx0 \# xx          q_6            xx0$ \\
Left shift (0,1,x) until we hit a $\#$.  \\
$xxxx0 \#             q_6            xxxx0$ \\
$xxxx0                q_7        \#  xxxx0$ \\
$xxxx                 q_7      0 \#  xxxx0$ \\
$xxxx0                q_1        \#  xxxx0$ \\
$xxxxx \#             q_2            xxxx0$ \\
$xxxxx \# x           q_4             xxx0$ \\
Right shift until we hit a 0. \\
$xxxxx \# xxxx        q_4                0$ \\
$xxxxx \# xxx         q_6               xx$ \\
Left shift (0,1,x) until we hit a $\#$. \\
$xxxxx \#             q_6            xxxxx$ \\
$xxxxx                q_7        \#  xxxxx$ \\
$xxxxx \#             q_1            xxxxx$ \\
$xxxxx \# x           q_8             xxxx$ \\
Right shift all the x's. \\
$xxxxx \# xxxxx       q_8  \Vtextvisiblespace    $ \\
$xxxxx \# xxxxx  \Vtextvisiblespace  q_{accept} \Vtextvisiblespace $ \\
$M_1$ enters into the $q_{accept}$ state, and as a result the input is accepted. \\

\pagebreak
And this one: $01101\#01100$ \\

Input is $01101\#01100$. Starting state is $q_1 01101\#01100$. \\
Running $01101\#01100$ on the machine $M_1$ results in the following config: \\

$                   q_1 01101 \# 01100$ \\
$x                  q_2  1101 \# 01100$ \\
Right shift (0,1) until we hit a $\#$   \\
$x1101              q_2       \# 01100$ \\
$x1101 \#           q_4          01100$ \\
$x1101              q_6       \# x1100$ \\
$x110               q_7     1 \# x1100$ \\
Left shift (0,1) until we hit a x       \\
$x                  q_7  1101 \# x1100$ \\
$x1                 q_1   101 \# x1100$ \\
$xx1                q_3    01 \# x1100$ \\
Right shift (0,1) until we hit a $\#$   \\
$xx101              q_3       \# x1100$ \\
$xx101 \#           q_5          x1100$ \\
$xx101 \# x         q_5           1100$ \\
$xx101 \#           q_6          xx100$ \\
$xx101              q_7       \# xx100$ \\
Left shift (0,1) until we hit a x       \\
$xx                 q_7   101 \# xx100$ \\
$xx1                q_1    01 \# xx100$ \\
$xxx0               q_3     1 \# xx100$ \\
$xxx01              q_3       \# xx100$ \\
$xxx01 \#           q_5          xx100$ \\
Right shift (x) until we hit a 1.       \\
$xxx01 \# xx        q_5            100$ \\
$xxx01 \# x         q_6           xx00$ \\
$xxx01 \#           q_6          xxx00$ \\
$xxx01              q_7       \# xxx00$ \\
Left shift (0,1) until we hit a x       \\
$xxx                q_7    01 \# xxx00$ \\
$xxx0               q_1     1 \# xxx00$ \\
$xxxx1              q_2       \# xxx00$ \\
$xxxx1 \#           q_4          xxx00$ \\
Right shift (x) until we hit a 0.       \\
$xxxx1 \# xxx       q_4             00$ \\
$xxxx1 \# xxxx      q_6              0$ \\
Left shift (0, 1, x) until we hit  $\#$ \\
$xxxx1 \#           q_6          xxxx0$ \\
$xxxx1              q_7      \#  xxxx0$ \\
$xxxx               q_7    1 \#  xxxx0$ \\
$xxxx1              q_1      \#  xxxx0$ \\
$xxxxx  \#          q_3          xxxx0$ \\
$xxxxx              q_5      \#  xxxx0$ \\
Right shift (x) until we hit a 1.
$xxxxx \#  xxxx     q_5              0$ \\
At this point the machine will go into the reject state. \\
The reason for this is that it cannot read a 1, or an x. \\
$xxxxx \#  xxxx0    q_{reject}        $ \\
$M_1$ enters into the $q_{reject}$ state, and as a result the input is rejected. \\

The reason for this failure is that at $q_5$, the machine expects a 1 but it gets \\
a 0. There are no paths for 0 at $q_5$, and as a result the machine will go into \\
the reject state as it is unable to continue. \\


%*********************************************************************************
% 3.8  b, c + other stuff
\pagebreak

3.8 Give implementation-level descriptions of Turing machines that decide the following
languages over the alphabet $\{0,1\}$. \\

b. $\{$w $\mid$ w contains twice as many 0s as 1s$\}$ \\

For input string $w$, we would do the following implementation: \\
\textbf{1.} We first scan the tape and mark the first 0 that has not been marked yet.
If we find no unmarked 0's, we then continue to $\#4$. \\

\textbf{2.} We move onto mark the next unmarked 0. If we do not find any on the tape,
we enter the reject state. Otherwise, we move back to the front of the tape. \\

\textbf{3.} We scan the tape and mark the first 1 which has not been marked yet. If there
is no unmarked 1, we enter the reject state. \\

\textbf{4.} We now move the head back to the front of the tape, and repeat $\#1$. \\

\textbf{5.} We move the head of the tape back to the front of the tape, and we then scan
the tape to see if we can find any unmarked 1's. If there are none, we enter the
accept state. Otherwise we enter the reject state. \\

c. $\{$w $\mid$ w does not contain twice as many 0s as 1s$\}$ \\

For input string $w$, we would do the following implementation: \\

\textbf{1.} We first scan the tape and mark the first 0 which has not yet been marked.
If we find no unmarked 0, we go to $\#4$. \\

\textbf{2.} We continue moving and mark the next unmarked 0. If we do not find any on
the tape, then we enter the accept state. Otherwise, we move the head of the tape back
to the front and we continue. \\

\textbf{3.} We scan the tape and mark the first 1 which has not yet been marked. If 
there are no unmarked 1's, we enter the accept state. \\

\textbf{4.} We move the head back to the front of the tape, and we repeat $#1$. \\

\textbf{5.} We move the head back to the front of the tape, and we scan the tape
to see if there are any unmarked 1's. If there are no unmarked 1's, we enter the 
reject state. Otherwise, we enter the accept state. \\

Plus: Draw the state diagram for Turning Machines 3.8b and 3.8c \\
3.8b \\

3.8c \\

For these machines draw the configurations for \\
3.8b: \\
010100. \\

010101. \\

3.8c: \\
000111. \\

000110. \\


%*********************************************************************************
% Modify machine M2 to recognize odd number of 0s and draw the state diagram. 
% Draw the configuration for 000 and 0000
\pagebreak

Modify machine M2 to recognize odd number of 0s and draw the state \\
diagram. \\

State diagram: \\

Draw the configuration for \\
000. \\

0000. \\

%*********************************************************************************
% Modify machine M2 to recognize even number of 0s and draw the state diagram. 
% Draw the configuration for 000 and 0000
\pagebreak

Modify machine M2 to recognize even number of 0s and draw the state \\
diagram. \\

State diagram: \\

Draw the configuration for \\
000. \\

0000. \\

\end{document}
